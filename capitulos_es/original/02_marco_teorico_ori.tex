\chapter{Marco Teórico}\label{ch:marco_teorico}

\section{Ciclones extratropicales en el Hemisferio Sur}\label{sec:mt_ciclones_sh}

\subsection{Definición, estructura y terminología básica}\label{subsec:mt_definicion}
Los ciclones extratropicales son sistemas de baja presión asociados a inestabilidad baroclínica, resultado de la interacción entre gradientes horizontales de temperatura, advecciones y la circulación en altura. Su estructura típica incluye frentes, un patrón asimétrico de ascensos y descensos, y una marcada dependencia del entorno sinóptico, lo que condiciona la distribución espacial de viento y precipitación en latitudes medias \citep{sinclair1997objective, simmonds2000mean}.
En el Atlántico Sur, además de los ciclones extratropicales, pueden presentarse ciclones subtropicales con características híbridas (p.\ ej., combinación de forzamiento baroclínico y rasgos de núcleo cálido), lo que obliga a precisar la terminología y el criterio de clasificación cuando se construyen climatologías regionales \citep{evans2012climatology}.

\subsection{Storm tracks del Hemisferio Sur y su variabilidad}\label{subsec:mt_storm_track}
Los ciclones se organizan en corredores preferenciales de tránsito (\textit{storm tracks}) modulados por la circulación media del oeste, la distribución océano--continente y la configuración de los jets. En el Hemisferio Sur, la mayor oceanidad favorece la persistencia del flujo zonal y condiciona la ubicación y forma del \textit{storm track}, con variabilidad interanual y decadal en su intensidad y posición \citep{hoskins2005new, inatsu2004zonal, simmonds2000variability}.
Esta organización es relevante para interpretar patrones espaciales recurrentes de impactos, ya que conecta la dinámica de gran escala con la frecuencia y rutas de los ciclones que afectan Sudamérica y el Atlántico Sur.

\section{Ciclogénesis en Sudamérica y Atlántico Sur}\label{sec:mt_ciclogenesis_sa}

\subsection{Rol de los Andes y la ciclogénesis a sotavento}\label{subsec:mt_andes_lee}
La Cordillera de los Andes actúa como una barrera orográfica que reorganiza el flujo incidente y favorece la ciclogénesis a sotavento mediante mecanismos de ajuste dinámico y generación de vorticidad. Este forzamiento, junto con la baroclinicidad regional, promueve zonas recurrentes de formación y posterior intensificación en el sector sudamericano \citep{gan1991surface, crespo2020potential}.
En consecuencia, la orografía no solo modula dónde se generan los ciclones, sino también cómo se organizan los gradientes y los flujos que controlan viento y precipitación asociados.

\subsection{Hotspots y regionalización (RG1--RG3)}\label{subsec:mt_rg}
La actividad ciclónica en el Atlántico Sur y áreas adyacentes presenta máximos regionales asociados a condiciones sinópticas y termodinámicas distintas. En un marco aplicado, la separación en regiones de ciclogénesis (p.\ ej., RG1, RG2 y RG3) permite tratar la climatología de impactos como poblaciones diferenciadas, reduciendo la mezcla de configuraciones atmosféricas heterogéneas \citep{reboita2010south, mendes2010climatology, dalanhese2023new}.
Esta regionalización es particularmente importante en el Atlántico Sur, donde pueden coexistir sistemas extratropicales y subtropicales; por ello, la delimitación regional y la definición del tipo de ciclón contribuyen a construir muestras más comparables \citep{evans2012climatology}.

\subsection{Ambientes de génesis y trayectorias típicas}\label{subsec:mt_ambientes}
Las propiedades del ambiente de génesis (baroclinicidad, jets, humedad disponible) condicionan la evolución del ciclón y la distribución de impactos. Estudios regionales han caracterizado entornos típicos de formación y rutas preferenciales en el Atlántico Sur, proporcionando base física para contrastes entre regiones de génesis \citep{gramcianinov2019properties, vera2002cold}.
Estas diferencias ambientales motivan que el análisis de viento y precipitación se condicione por región, evitando interpretar como “variabilidad interna” lo que en realidad responde a forzamientos sinópticos distintos.

\section{Identificación y tracking objetivo de ciclones}\label{sec:mt_tracking}

\subsection{Criterios de detección: presión vs vorticidad (850 hPa)}\label{subsec:mt_presion_vort}
Los métodos objetivos de detección difieren según el campo usado para definir el centro del ciclón. Los enfoques basados en presión mínima (SLP) capturan la señal sinóptica clásica, mientras que los basados en vorticidad relativa en niveles bajos (p.\ ej., 850 hPa) tienden a mejorar la identificación de fases incipientes y estructuras compactas, particularmente relevantes en el Hemisferio Sur \citep{sinclair1997objective, hodges2011comparison}.
En términos aplicados, el campo de detección condiciona la muestra final (conteos, duración, intensidad aparente) y, por tanto, influye en cualquier estadística de extremos asociada.

\subsection{Métricas del ciclón: intensidad, tamaño, velocidad y duración}\label{subsec:mt_metricas}
La caracterización del ciclón requiere métricas que describan su intensidad (p.\ ej., circulación o profundidad), su tamaño (radio efectivo/escala), su velocidad de traslación y su duración. La evolución de estas métricas a lo largo del ciclo de vida permite separar cambios estructurales de cambios meramente geométricos del centro del sistema \citep{simmonds2000size, sinclair1997objective}.
Estas métricas también permiten conectar el ciclo de vida con la evolución espacial de los máximos de viento y precipitación, que no siempre se ubican en el centro dinámico.

\subsection{Sensibilidad a reanálisis y consistencia entre bases}\label{subsec:mt_reanalisis}
La estadística de ciclones depende del reanálisis y del algoritmo de tracking, por diferencias en resolución, asimilación y representación de gradientes. Comparaciones entre reanálisis muestran variaciones en conteos, intensidad y trayectorias, por lo que es necesario explicitar el producto usado y su compatibilidad con el objetivo del estudio \citep{hodges2011comparison, dalanhese2023new}.
En consecuencia, la interpretación de climatologías condicionadas por fase y región debe considerar que parte de la dispersión estadística puede provenir del propio sistema de datos y del método de detección.

\section{Ciclo de vida y fases del ciclón}\label{sec:mt_fases}

\subsection{Ciclo de vida y evolución temporal de propiedades}\label{subsec:mt_ciclo_vida}
El ciclón evoluciona mediante etapas en las que cambian su intensidad, su tamaño y la organización espacial de sus campos asociados. Resultados previos muestran que el tamaño y otras propiedades no son constantes durante la vida del sistema, lo que respalda enfoques que condicionan el análisis por fase \citep{simmonds2000size}.
Por ello, la “misma” intensidad central puede corresponder a estructuras y huellas de impacto distintas según el estado de vida del ciclón.

\subsection{Definición operativa de fases: incipiente, intensificación, madurez y decaimiento}\label{subsec:mt_def_fases}
Una división operativa del ciclo de vida en cuatro fases facilita relacionar la dinámica del ciclón con la evolución de impactos en superficie. En enfoques automatizados, las fases pueden definirse a partir de tasas de cambio y umbrales de variables dinámicas (p.\ ej., vorticidad), permitiendo segmentación consistente en grandes climatologías \citep{coutodesouza2024new}.
Este tipo de segmentación es especialmente útil cuando el objetivo es comparar patrones espaciales de máximos de variables de superficie bajo estados dinámicos equivalentes.

\subsection{No linealidad y transiciones de fase}\label{subsec:mt_nolinealidad}
La progresión entre fases no necesariamente es estrictamente lineal; algunos ciclones pueden transitar entre etapas o reintensificarse, por lo que conviene considerar la fase como un estado dinámico identificado objetivamente, más que como una secuencia fija. Esta perspectiva es clave para interpretar distribuciones de extremos condicionadas por fase \citep{coutodesouza2024new}.
En términos de impactos, esta no linealidad implica que los máximos de viento y precipitación pueden reorganizarse espacialmente de forma no monótona a lo largo del tiempo de vida.

\section{Viento y precipitación asociados a ciclones y sus extremos}\label{sec:mt_impactos}

\subsection{Organización espacial de viento y precipitación en ciclones}\label{subsec:mt_org_espacial}
Los campos de viento y precipitación alrededor de un ciclón son asimétricos y dependen de la estructura frontal, el acoplamiento con el jet y el transporte de humedad. En el Atlántico Sur, la configuración sinóptica y la disponibilidad de humedad modulan la extensión y el posicionamiento de las bandas de precipitación, así como los máximos de viento en superficie \citep{gramcianinov2019properties, reboita2018extratropical}.
Esta asimetría justifica el análisis espacial (y no solo “central”) de variables de superficie cuando se busca describir climatología e impactos extremos.

\subsection{Extremos en superficie y desplazamiento respecto al centro}\label{subsec:mt_extremos}
Los máximos de viento y precipitación no necesariamente coinciden con el centro del ciclón, y su localización puede variar con la fase y el ambiente. Esta separación entre el centro dinámico y los máximos de impacto justifica analizar la huella espacial completa de las variables, en lugar de evaluar la intensidad únicamente con una métrica central \citep{sinclair2023relationship, padilhareinke2026characterization}.
Bajo esta lógica, los estadísticos de máximos y su distancia al centro del ciclón permiten describir patrones robustos de impacto que no se capturan con indicadores centralizados.

\subsection{Justificación del análisis por región y fase}\label{subsec:mt_justificacion}
Dado que las regiones de ciclogénesis presentan configuraciones atmosféricas distintas, la comparación de extremos requiere condicionar el análisis por región para evitar mezclar poblaciones con dinámicas diferentes. Del mismo modo, el condicionamiento por fase permite capturar cambios sistemáticos en la estructura espacial de viento y precipitación a lo largo de la vida del ciclón, coherente con el objetivo de caracterizar climatología e impactos extremos \citep{dalanhese2023new, coutodesouza2024new}.
En el Atlántico Sur, además, la existencia de ciclones subtropicales refuerza la necesidad de una definición clara del tipo de ciclón y de criterios consistentes de muestreo cuando se interpretan impactos en superficie \citep{evans2012climatology}.

\section{Enfoques multivariados para patrones espacio-temporales}\label{sec:mt_eof}

\subsection{EOF/PCA: concepto y utilidad en meteorología}\label{subsec:mt_eof_concepto}
El análisis EOF/PCA permite representar un campo espacio-temporal mediante un conjunto reducido de patrones ortogonales y sus series temporales asociadas (componentes principales), maximizando la varianza explicada y reduciendo dimensionalidad. En meteorología, esta familia de métodos se usa para (i) identificar modos dominantes de variabilidad sinóptica y (ii) construir espacios reducidos para clasificación objetiva o comparación entre conjuntos de datos.
%
% Cornér et al. (NHESS 2025, doi:10.5194/nhess-25-207-2025):
% "performing a sparse principal component analysis on the set of measures."
% (paper: Classification of North Atlantic and European extratropical cyclones using multiple measures of intensity)
\citep{corner2025classification}
%
% Liang et al. (2018):
% "The multivariate EOF (MEOF) method, a variant of the archetypal EOF method..."
\citep{liang2018multivariate}

\subsection{EOF por fase y por región: interpretación física}\label{subsec:mt_eof_fase_region}
Aplicar EOF/PCA por fase y por región permite extraer patrones dominantes condicionados a estados dinámicos comparables y a ambientes de génesis más homogéneos, evitando que la mezcla de fases o regiones diluya señales estructurales. En particular, extensiones multivariadas (p.\ ej., MEOF) permiten incorporar más de una variable para describir covariabilidad acoplada y resumir estructuras consistentes con mecanismos físicos dominantes \citep{liang2018multivariate}.

\subsection{Distribuciones de extremos (KDE/histogramas) como complemento}\label{subsec:mt_kde}
Además de resumir patrones mediante EOF/PCA, es útil caracterizar la distribución de máximos (y su localización relativa al centro) con histogramas o KDE, lo que entrega una descripción probabilística de magnitudes extremas y su dispersión. En estudios de ciclones, el KDE se ha aplicado tanto para densidades (p.\ ej., densidad de trayectorias/fases) como para distribuciones conjuntas entre intensidad y distancia al centro, o para delimitar regímenes en espacios de parámetros.
%
% Souza et al. (CycloPhaser, 2024, doi:10.1002/joc.8539):
% "The cyclone density is computed using the kernel density estimation (KDE) method..."
\citep{coutodesouza2024new}
%
% Priestley & Catto (GRL, 2022, doi:10.1029/2021GL096708):
% "a Gaussian kernel density estimation of the maximum wind speed and its distance from cyclone center..."
\citep{priestley2022improved}
%
% Han & Ullrich (JGR-A, 2025, doi:10.1029/2024JD041287):
% "we plot the kernel density estimate (KDE) on the RH100MAX-DEEPSHEAR coordinate..."
\citep{han2025syclops}

% Nota práctica:
% - Si tu objetivo es “máximos por fase y región + distancia al centro”, Priestley2022 te sirve como antecedente directo
%   (KDE en espacio {Vmax, distancia}), y Souza2024 como antecedente directo para KDE de densidad por fase (CycloPhaser).
% - Cornér2025 te sirve como antecedente metodológico (PCA/sPCA) para reducir medidas/atributos y clasificar ciclones en un
%   espacio de baja dimensión (idea transferible a métricas de impacto o a descriptores por fase).