\chapter{Introducción} \label{ch:introduccion}

% --- PÁRRAFO 1: CONTEXTO CLIMÁTICO (IPCC) Y RIESGO ---
En el contexto del cambio climático antropogénico, la comunidad científica internacional ha puesto un énfasis renovado en comprender la dinámica de los eventos meteorológicos extremos. Según el resumen del Sexto Informe de Evaluación del IPCC presentado por \citet{masson2022tendencias}, existe una evidencia inequívoca de que la frecuencia e intensidad de las precipitaciones extremas están aumentando en regiones específicas del globo, destacándose el Sudeste de América del Sur (SES) como una de las zonas más afectadas. Esta tendencia observada impone desafíos críticos para la gestión del riesgo de desastres. En las latitudes medias, estos extremos hidrometeorológicos están dinámicamente acoplados a los ciclones extratropicales. Estos sistemas de baja presión son los responsables directos de los eventos de vientos destructivos, marejadas ciclónicas severas y tormentas intensas que impactan regularmente las costas de Argentina, Uruguay y el sur de Brasil, generando impactos socioeconómicos significativos \citep{reboita2018extratropical}.

% --- PÁRRAFO 2: LA MÁQUINA ATMOSFÉRICA DEL HEMISFERIO SUR ---
Para comprender la génesis de estos eventos en Sudamérica, es necesario situarlos primero en el contexto de la circulación general del Hemisferio Sur. A diferencia de su contraparte boreal, donde la alternancia zonal de grandes masas continentales y cuencas oceánicas genera ondas estacionarias que perturban el flujo medio, el Hemisferio Sur presenta una configuración geográfica dominada por la ``oceanidad''. La existencia de un cinturón oceánico casi continuo alrededor de la Antártida permite que los vientos del oeste (\textit{westerlies}) fluyan con una persistencia e intensidad notables a lo largo de todo el año \citep{simmonds2000mean}. Dentro de este flujo zonal, los ciclones extratropicales se organizan en un corredor de tormentas o \textit{storm track} bien definido. Estudios seminales han caracterizado este corredor como una estructura en espiral asimétrica que nace en las latitudes medias de los océanos Atlántico, Índico y Pacífico, y se curva progresivamente hacia el sureste \citep{hoskins2005new}. Esta trayectoria espiralada transporta los sistemas hacia las altas latitudes, convergiendo finalmente en la costa de la Antártida, una región de alta densidad de disipación frecuentemente descrita en la literatura como el ``cementerio de ciclones'' \citep{simmonds1999southern}.

% --- PÁRRAFO 3: EL ROL DE LOS ANDES ---
Dentro de este vasto dominio hemisférico, el continente sudamericano introduce una perturbación fundamental. La Cordillera de los Andes, que se extiende latitudinalmente como una barrera orográfica formidable, interactúa mecánicamente con el flujo incidente de los oestes. Esta interacción fuerza el ascenso del aire a barlovento y genera una compresión de la columna de vorticidad, seguida de un estiramiento vertical a sotavento. Por el principio de conservación de la vorticidad potencial, este proceso induce la formación de vaguadas y centros de baja presión al este de la cordillera, un mecanismo conocido como ciclogénesis orográfica o \textit{lee cyclogenesis} \citep{inatsu2004zonal, gan1991surface}. Debido a este forzante topográfico constante, la costa este de Sudamérica y el Atlántico Sudoccidental se configuran no solo como una zona de tránsito, sino como una de las regiones de ciclogénesis más activas y con tasas de desarrollo más rápidas del planeta \citep{sinclair1995climatology}.

% --- PÁRRAFO 4: ZONIFICACIÓN (RG1, RG2, RG3) ---
La climatología de esta región no es homogénea, sino que presenta una clara estructuración latitudinal. Investigaciones exhaustivas han permitido identificar y consolidar la existencia de tres focos principales de acción o ``hotspots'' de ciclogénesis: la región subtropical ubicada frente a las costas del sureste de Brasil (RG1), la zona de transición sobre el Río de la Plata y Uruguay (RG2), y la región patagónica en el sur de Argentina (RG3) \citep{reboita2010south}. Estudios comparativos utilizando diferentes reanálisis han confirmado que, si bien la frecuencia absoluta de sistemas puede variar según el método de detección, la ubicación espacial de estos tres centros de acción es robusta y persistente en el tiempo \citep{mendes2010climatology}. La distinción entre estas regiones no es meramente geográfica, sino que responde a regímenes dinámicos diferenciados que modulan la naturaleza y el impacto de los ciclones generados.

% --- PÁRRAFO 5: DINÁMICA DE ONDAS (SUBTROPICAL VS POLAR) ---
La variabilidad de la actividad ciclónica en estos focos es controlada por la estructura de la corriente en chorro en altura. \citet{vera2002cold} demostraron que las perturbaciones de escala sinóptica en Sudamérica se propagan a través de dos guías de onda (\textit{waveguides}) principales: una guía de onda polar, asociada al jet polar alrededor de los 60°S, y una guía de onda subtropical, asociada al jet subtropical alrededor de los 30°S. Los ciclones formados en la región patagónica (RG3) están dinámicamente vinculados a la guía de onda polar y son impulsados primariamente por la inestabilidad baroclínica profunda derivada de la cizalladura vertical del viento. En contraste, los sistemas que afectan al sureste de Brasil (RG1) están asociados a la guía de onda subtropical, donde la interacción entre vaguadas de altura y la superficie juega un rol más complejo y variable.

% --- PÁRRAFO 6: TERMODINÁMICA (CALOR LATENTE) ---
En las latitudes subtropicales y medias (RG1 y RG2), un segundo forzante físico cobra una importancia crítica: la termodinámica del océano. Esta región alberga la Confluencia Brasil-Malvinas, una de las zonas oceánicas más energéticas del mundo, caracterizada por fuertes gradientes meridionales de Temperatura Superficial del Mar (TSM). Investigaciones recientes sugieren que estos gradientes oceánicos actúan como una fuente de energía diabática, inyectando flujos intensos de calor latente y sensible hacia la atmósfera baja \citep{gramcianinov2019properties}. Esta inyección de energía es capaz de desestabilizar la columna atmosférica y potenciar el desarrollo ciclónico, un mecanismo fundamental para explicar la ocurrencia de ciclones explosivos o ``bombas'' meteorológicas en latitudes donde el forzante dinámico puro podría ser insuficiente \citep{machado2020influence}.

% --- PÁRRAFO 7: LIMITACIONES DE DATOS ANTIGUOS ---
A pesar de la riqueza y complejidad de estos mecanismos, nuestra capacidad histórica para caracterizarlos estuvo severamente limitada por la tecnología de observación y análisis. Durante décadas, los estudios climatológicos se basaron en reanálisis de baja resolución espacial y en algoritmos de seguimiento fundamentados en la presión mínima al nivel del mar (MSLP). Sin embargo, se ha demostrado teóricamente y empíricamente que el uso de la presión introduce un sesgo sistemático: favorece la detección de sistemas grandes, lentos y profundos, mientras que tiende a omitir o subestimar aquellos ciclones de mesoescala, rápidos y móviles, que suelen estar ``enmascarados'' por el fuerte flujo medio de los oestes \citep{sinclair1994objective, sinclair1997objective}. Esta limitación implicaba que los ciclones más violentos y de rápida evolución a menudo quedaban invisibilizados en las estadísticas climáticas tradicionales.

% --- PÁRRAFO 8: LA REVOLUCIÓN ERA5 ---
La reciente disponibilidad del reanálisis ERA5 del Centro Europeo (ECMWF) ha marcado un cambio de paradigma en la meteorología sinóptica. Con una resolución horizontal de aproximadamente 31 km y un sistema de asimilación de datos 4D-Var avanzado, ERA5 ofrece una capacidad sin precedentes para resolver la estructura fina de la atmósfera \citep{hersbach2020era5}. Comparaciones directas han demostrado que ERA5 captura la intensidad de los vientos superficiales y los núcleos de vorticidad con una fidelidad superior a cualquier producto anterior, revelando que la frecuencia de ciclones explosivos en el Atlántico Sur es mayor de lo que se estimaba previamente \citep{dalanhese2023new}. Estudios de sensibilidad confirman que esta alta resolución es un requisito indispensable para representar correctamente la asimetría del campo de viento y los extremos de precipitación asociados a la fase de madurez del ciclón \citep{priestley2022improved}.

% --- PÁRRAFO 9: EL VACÍO CIENTÍFICO (CICLO DE VIDA) ---
A pesar de estos avances instrumentales, persiste un vacío significativo en la literatura. La gran mayoría de los estudios existentes se han centrado en climatologías de frecuencia media o en análisis de casos individuales, dejando de lado una caracterización sistemática de la evolución física de los eventos extremos. No basta con saber dónde se forman los ciclones; es necesario comprender cómo evoluciona su estructura termodinámica y cinemática a lo largo de su ciclo de vida. El enfoque moderno propone diseccionar la vida del sistema en fases discretas —incipencia, intensificación, madurez y decaimiento— para entender en qué momento y lugar se maximizan los impactos \citep{coutodesouza2024new}. Esta perspectiva es crucial para diferenciar entre un ciclón ``promedio'' y aquellos pertenecientes al percentil superior (``Top 10\%''), cuya dinámica puede obedecer a procesos de intensificación no lineales y acoples rápidos \citep{andrade2024composite}.

% --- PÁRRAFO 10: OBJETIVO DEL ESTUDIO (CORREGIDO) ---
Este estudio busca llenar este vacío de conocimiento, aprovechando la robustez de los datos de ERA5 para realizar una caracterización integral de los ciclones extratropicales extremos en el Atlántico Sur. El objetivo central es analizar cómo varía la estructura espacial de las variables de impacto en superficie (viento y precipitación) a través de las diferentes fases del ciclo de vida, y determinar cómo estas características difieren según la región de génesis y los mecanismos de intensificación involucrados. Al vincular la dinámica atmosférica de gran escala con la huella de impacto en superficie, se pretende ofrecer nuevas herramientas conceptuales para la previsión y mitigación de desastres en una región que, tal como proyectan \citet{dejesus2021multimodel}, es cada vez más vulnerable a los extremos climáticos.