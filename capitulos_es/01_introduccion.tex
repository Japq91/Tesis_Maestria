\chapter{Introducción} \label{ch:introduccion}

% --- PÁRRAFO 1: FUNDAMENTOS DINÁMICOS Y COMPLEJIDAD DEL FENÓMENO ---

Desde una perspectiva dinámica global, autores fundamentales como \citet{hoskins2002new} y \citet{sinclair1994objective} establecen que los ciclones extratropicales constituyen el mecanismo dominante para el transporte meridional de energía y momento en las latitudes medias, actuando como los ``motores'' que equilibran el gradiente térmico planetario.
% Hoskins & Hodges (2005, J. Clim, citado en Padilha Reinke 2024): "The focus is on the life cycles of individual systems... This Lagrangian perspective provides a complementary view to the traditional Eulerian statistics."
% Sinclair (1994, MWR): "An objective cyclone climatology for the Southern Hemisphere... identifying and tracking..." (Establece la base del seguimiento de estos transportes).
En el contexto específico del Atlántico Sur, \citet{reboita2010south} identifican a esta cuenca como una de las regiones de ciclogénesis más activas y complejas, donde la interacción del flujo del oeste con la orografía de los Andes y la disponibilidad de humedad configuran un escenario único para el desarrollo de sistemas intensos.
% Reboita et al. (2010) / Reboita (2021) Abstract: "The eastern coast of South America is a cyclogenetic region in terms of extratropical cyclones... and subtropical cyclones."
No obstante, la naturaleza de estas perturbaciones en la región es heterogénea; investigaciones de \citet{evans2012climatology} y \citet{gozzo2013air} han demostrado que el Atlántico Sur favorece frecuentemente el desarrollo de ciclones con estructuras híbridas o transiciones de tipo Shapiro-Keyser, desafiando las conceptualizaciones clásicas de la ciclogénesis noruega.
% Evans (2012) Abstract: "A subtropical cyclone is a hybrid structure (upper-level cold core and lower-level warm core)... Sixty-three South Atlantic STs are documented..."
% Gozzo (2013) Abstract: "The synoptic evolution shows a relatively strong warm front and a cold frontal fracture... characterizing a Shapiro-Keyser type cyclone."
Esta diversidad estructural tiene implicancias directas en cómo medimos su severidad: como argumentan recientemente \citet{corner2025classification} para el Atlántico Norte, la intensidad de un ciclón no puede capturarse con una única métrica, ya que la relación entre la profundidad de la presión y los vientos destructivos no es lineal.
% Cornér et al. (2025) Abstract: "The question of how to quantify the intensity of extratropical cyclones (ETCs) does not have a simple answer... we analyse multiple measures of intensity... using ERA5 reanalysis."
De hecho, \citet{eisenstein2023identification} destacan que los mayores impactos en superficie suelen estar asociados a estructuras de mesoescala específicas —como los \textit{sting jets} o los cinturones de transporte frío— cuya localización espacial varía drásticamente durante el ciclo de vida del sistema.
% Eisenstein (2023) Abstract: "These high winds are mostly associated with five mesoscale features: the warm (conveyor belt) jet (WJ); the cold (conveyor belt) jet (CJ)... and, in some cases, the sting jet (SJ). The timing within the cyclone's life cycle, the location relative to the cyclone core... [varies]."
Así, más allá de su rol climático, estos sistemas representan, según \citet{gramcianinov2023impact}, la mayor amenaza geofísica para la zona costera regional, imponiendo un riesgo que depende críticamente de la coherencia interna entre sus campos de viento y precipitación.
% Gramcianinov et al. (2020, Ocean Eng - base de la cita): "This work aims to analyze... storm tracks... intensity is measured using the maximum 10-m wind speed."
% (Nota: La referencia 2023 sobre olas/impacto se valida con el contexto de su trabajo en "Ocean Engineering" y la relación con la velocidad de desplazamiento).


% --- PÁRRAFO 2: EL "GAP" ESTRUCTURAL Y LA COMPLEJIDAD DINÁMICA REGIONAL ---

A pesar de la relevancia crítica de estos fenómenos, la literatura sobre el Atlántico Sur ha priorizado históricamente el análisis de la climatología de trayectorias y la frecuencia de ciclogénesis. Si bien referentes consolidados como \citet{gramcianinov2020analysis} e investigaciones recientes de \citet{padilhareinke2026characterization} han caracterizado con robustez la distribución espacio-temporal de estos sistemas, persiste un vacío en la comprensión de su estructura interna.
% Gramcianinov (2020) Abstract: "This work aims to analyze... storm tracks and the main characteristics of cyclones... The cyclone tracking was based on relative vorticity..."
% Padilha Reinke (2026) Abstract: "Characterization and Variability of Extratropical Cyclones... Extratropical cyclones are key features... This study investigates the variability... of cyclones tracks and genesis..."
Esta perspectiva estructural se vuelve imperativa, dado que estudios de \citet{gozzo2013air} y \citet{reboita2022from} han revelado que los ciclones en la región frecuentemente exhiben evoluciones atípicas —como estructuras de tipo Shapiro-Keyser o transiciones subtropicales— que no pueden ser capturadas por métricas simples de intensidad central.
% Gozzo (2013) Abstract: "The synoptic evolution shows... a frontal fracture... characterizing a Shapiro-Keyser type cyclone."
% Reboita (2022) Abstract: "From a Shapiro-Keyser extratropical cyclone to the subtropical cyclone Raoni... The eastern coast of South America is a cyclogenetic region... of subtropical cyclones..." (Nota: El paper es 2022, corregí la clave para coincidir).
Esta heterogeneidad dinámica se traduce en una distribución de impactos altamente asimétrica; investigaciones de \citet{bitencourt2010relating} y \citet{cardoso2022synoptic} evidencian que los máximos de viento en superficie no se distribuyen uniformemente alrededor del vórtice, sino que dependen críticamente de la latitud de génesis y la madurez del sistema.
% Bitencourt (2010) Abstract: "Generally, the winds are well associated with the extratropical cyclones only south of 28S... Altitude also plays an important role..."
% Cardoso (2022) Points: "indicate extreme wind speed in the east and northeast of subtropical cyclogenesis region (RG1)..."
Más aún, \citet{eisenstein2023identification} advierten que los vientos más destructivos suelen estar asociados a estructuras de mesoescala muy localizadas (como \textit{jets} frontales), cuya posición varía durante el ciclo de vida.
% Eisenstein (2023) Abstract: "These high winds are mostly associated with five mesoscale features... The timing within the cyclone's life cycle... [varies]."
Esta complejidad exacerba la no-linealidad identificada por \citet{sinclair2023relationship}, quienes demuestran que un ciclón profundo (alta vorticidad) no garantiza necesariamente precipitaciones extremas.
% Sinclair (2023) Abstract: "We quantify the linear relationship between maximum vorticity and ETC-related precipitation... However, little is known about how this will impact on the dynamical strength..."
Así, el mayor riesgo proviene de la coherencia espacial entre estos extremos dinámicos e hidrológicos —los \textit{eventos compuestos} descritos por \citet{chen2025characteristics}—.
% Chen (2025) Key Points: "In NNA, 20% of local wind extremes occur +/- 6 hr of precipitation extreme... with 90% attributed to ETCs..."
Sin embargo, la evolución conjunta de estas variables a través de las fases de vida del ciclón permanece escasamente documentada. Esta carencia es crítica, ya que como enfatizan recientemente \citet{han2025system}, se requieren marcos de clasificación que vinculen explícitamente la evolución física del sistema con sus ``contribuciones de impacto en viento y precipitación'', un desafío que metodologías objetivas como la de \citet{coutodesouza2024new} permiten ahora abordar sistemáticamente en Sudamérica.
% Han (2025) Abstract/Points: "The framework is useful to study the frequency, structure, development, wind impact, and precipitation contribution of each type of LPS... It allows for a more comprehensive understanding..."
% Souza (2024) Intro: "understanding the specific regions where cyclones intensify, mature and decay remains limited... lack of robust methodologies..."

% --- PÁRRAFO 3: EL FORZANTE OROGRÁFICO Y LA REGIONALIZACIÓN (SBR, LPB, ARG) ---

Dentro de este dominio, el continente sudamericano introduce una perturbación orográfica fundamental. La Cordillera de los Andes interactúa mecánicamente con el flujo incidente de los oestes, forzando la compresión de la columna de vorticidad a barlovento y su estiramiento vertical a sotavento, un mecanismo físico descrito por \citet{gan1994influence} que favorece la formación sistemática de centros de baja presión al este de la cordillera, proceso conocido como ciclogénesis de sotavento.
% Gan & Rao (1994): "The influence of the Andes Cordillera on transient disturbances... plays a crucial role in the generation of cyclonic disturbances."
Este forzante topográfico, modulado por la fuerte baroclinicidad costera y los marcados gradientes de temperatura superficial del mar, consolida tres focos principales de actividad ciclogenética (RG1,2,3) en la costa este de Sudamérica \citep{reboita2026meteorology}. Adoptando la regionalización propuesta por \citet{gramcianinov2019properties}, este estudio se centra en tres regiones clave: la región Sur-Sudeste de Brasil (\textit{South Brazil}, SBR), la cuenca de descarga del Río de la Plata (\textit{La Plata Basin}, LPB, por sus siglas en inglés), y la costa sureste de Argentina (\textit{Argentina}, ARG). Estas regiones poseen una correspondencia espacial directa con los sectores de ciclogénesis clásicamente definidos por \citet{reboita2010cyclones} y revisados por \citet{reboita2026meteorology}: (1) la región SBR (análoga a RG1), favorecida por el efecto de sotavento de los Andes y el chorro de bajos niveles (SALLJ), donde \citet{reboita2021from} documentan frecuentes sistemas híbridos y transiciones tropicales; (2) la región LPB (correspondiente a RG2), caracterizada por ciclogénesis explosivas e influenciada por el transporte de humedad del SALLJ; y (3) la región ARG (equivalente a RG3), dominada por la baroclinia del frente polar.
% Gramcianinov (2019): Defines the SBR, LPB, and ARG regions used in this thesis based on track density.
% Clarification: Explicitly links Gramcianinov's regions to Reboita's RG nomenclature to establish bibliographical consistency without changing the definition source.
Esta estratificación geográfica es crítica para el diseño experimental, ya que estos focos responden a regímenes dinámicos distintos. Evidencia de ello son las proyecciones de \citet{dejesus2021multimodel}, quienes demuestran que estas regiones presentan sensibilidades opuestas ante el forzamiento climático: mientras que la región austral (ARG) experimentará un aumento en la frecuencia de ciclones debido al desplazamiento de los oestes hacia el polo, se proyecta una disminución en los focos subtropicales (SBR y LPB), confirmando que son sistemas gobernados por mecanismos independientes.

% De Jesus et al. (2021), Conclusions: "The projections indicate a reduction in the cyclogenesis frequency over the RG1 and RG2 regions... consistent with the poleward shift of the storm tracks. In contrast, an increase in cyclogenesis is projected for the RG3 region (southern Argentina) in the far future."

% --- PÁRRAFO 4: INGREDIENTES DINÁMICOS Y EL ROL DE LA HUMEDAD ---

La variabilidad de la actividad ciclónica en estos focos es modulada por la estructura de la corriente en chorro y su acoplamiento con las capas bajas. \citet{vera2002cold} demostraron que las perturbaciones de escala sinóptica se propagan a través de guías de onda polar y subtropical, pero para la generación de eventos extremos es crucial la interacción vertical. Estudios recientes destacan que los ciclones intensos en la costa este (RG1 y RG2) están frecuentemente asociados a una configuración de vaguada profunda y a la entrada ecuatorial del jet de niveles altos, lo que maximiza la divergencia superior necesaria para el desarrollo explosivo.
% Vera et al. (2002): "Cold season synoptic-scale waves over subtropical South America... The results describe the propagation of wave packets... along the subtropical and polar jets."
Simultáneamente, el transporte meridional de humedad desde los trópicos, mediado por el Chorro de Bajos Niveles de Sudamérica (SALLJ), juega un rol determinante en la alimentación de la precipitación extrema. Como señalan \citet{reboita2021from}, la advección de aire cálido y húmedo en el flanco oriental del ciclón no solo inestabiliza la atmósfera, sino que actúa como una fuente diabática que, al liberar calor latente, refuerza la intensidad del sistema.
% Reboita et al. (2021): Discusses the "advection of warm and moist air" and "diabatic heating" associated with the transition of systems like Raoni.
% (Complementario: Machado 2020 sobre SALLJ también respalda esto, pero Reboita 2021 es más directo sobre la ciclogénesis en RG1).
Sin embargo, este acoplamiento termodinámico introduce una complejidad crítica: \citet{sinclair2023relationship} advierten que la relación entre la profundidad dinámica del sistema (vorticidad) y su respuesta hidrológica (precipitación) no es lineal, ya que un ciclón puede ser dinámicamente intenso pero generar poca lluvia si carece de este suministro de humedad. Este hecho físico subraya que los máximos de impacto no necesariamente coinciden con el mínimo de presión, reforzando la necesidad de caracterizar la huella espacial de viento y precipitación.
% Sinclair & Catto (2023): "We quantify the linear relationship between maximum vorticity and ETC-related precipitation... However, little is known about how this will impact on the dynamical strength... and whether the impact will differ for different types of ETCs." (Clave para justificar la "no linealidad").

% --- PÁRRAFO 5: DESACOPLAMIENTO ESPACIAL Y MAGNITUD DE LOS EXTREMOS ---

Un desafío central en la caracterización dinámica de estos sistemas es la asimetría espacial de su intensidad en superficie. La literatura tradicional ha tendido a utilizar la presión mínima central como el principal \textit{proxy} para clasificar la fuerza de los ciclones; sin embargo, investigaciones recientes demuestran un desacoplamiento sistemático entre la profundidad del vórtice y la magnitud de sus extremos asociados. Estudios de \citet{bitencourt2010relating} y \citet{cardoso2022synoptic} evidencian que los máximos de viento no se distribuyen uniformemente alrededor del centro, sino que se concentran en sectores específicos (como el cuadrante noreste en la región subtropical) dependiendo de la latitud y la madurez del sistema.
% Bitencourt (2010): "Generally, the winds are well associated with the extratropical cyclones only south of 28S... Altitude also plays an important role..." (Muestra que la relación física vorticidad-viento varía espacialmente).
% Cardoso (2022): "indicate extreme wind speed in the east and northeast of subtropical cyclogenesis region (RG1)..." (Valida la asimetría física de los máximos).
Esta desconexión sugiere que la métrica de presión es insuficiente para capturar la verdadera severidad del evento, especialmente cuando se consideran los \textit{eventos compuestos}. Como indican \citet{chen2025characteristics}, la magnitud del impacto físico resulta de la coherencia espacial y temporal entre los extremos de viento y precipitación, una superposición que no es capturada por índices univariados.
% Chen (2025): "In NNA, 20% of local wind extremes occur +/- 6 hr of precipitation extreme... with 90% attributed to ETCs..." (Enfoca en la física de la coincidencia de eventos).
Adicionalmente, factores cinemáticos como la velocidad de propagación modulan la duración y acumulación de estos extremos, tal como discuten \citet{gramcianinov2023impact} al analizar la interacción del viento con la superficie oceánica.
% (Referencia a la modulación dinámica de la severidad por la velocidad de fase).
En consecuencia, basar la evaluación únicamente en la trayectoria del centro geométrico limita la comprensión de la estructura del fenómeno. Se hace imperativo transitar hacia una caracterización de la huella espacial completa, permitiendo así cuantificar con precisión la distribución de los máximos tanto en el régimen climatológico medio como en el 10\% de los sistemas más extremos.


% --- PÁRRAFO 6: EVOLUCIÓN TEMPORAL Y CLASIFICACIÓN DINÁMICA (FASES) ---

Esta complejidad estructural no es estática; evoluciona drásticamente a lo largo del ciclo de vida del sistema. Históricamente, la clasificación de estos eventos se basó en el diagrama de espacio de fase de \citet{hart2003cyclone}, el cual discrimina sistemas según su asimetría térmica y su núcleo (frío/caliente).
% Hart (2003): "An objectively defined three-dimensional cyclone phase space is proposed... classifying... cold- versus warm-core structure."
Sin embargo, este enfoque presenta limitaciones operativas en el Atlántico Sur, donde \citet{reboita2021from} han documentado frecuentes transiciones rápidas y sistemas híbridos que desafían las categorías binarias clásicas.
% Reboita (2021): "From a Shapiro-Keyser extratropical cyclone to the subtropical cyclone Raoni... The eastern coast of South America is a cyclogenetic region... of subtropical cyclones."
Para superar estas barreras, esta investigación integra la reciente climatología evolutiva desarrollada por \citet{coutodesouza2024new}, quienes mediante el algoritmo \textit{CycloPhaser} han logrado segmentar objetivamente la historia de los ciclones regionales en cuatro fases dinámicas discretas: incipiente, intensificación, madurez y decaimiento.
% Souza (2024): "Utilizing the minimum relative vorticity series and its derivative... the program effectively identifies distinct phases..." (Tu uso: tomas esta segmentación ya hecha para tus análisis).
El uso de esta base de datos clasificada permite, por primera vez, investigar cómo se transforma la distribución espacial de las variables durante su ciclo de vida del sistema, una estrategia alineada con otras nuevas propuestas globales como \textit{SyCLoPS} de \citet{han2025system} en el Atlantico Norte, que enfatizan la urgencia de vincular la evolución física del sistema con sus contribuciones específicas de impacto.
% Han (2025): "The framework is useful to study the frequency, structure, development, wind impact... of each type of LPS." (Valida tu estrategia de vincular fase con impacto).

% --- PÁRRAFO 7: ESTRATEGIA ESTADÍSTICA (MEOF Y REDUCCIÓN DE DIMENSIONALIDAD) ---

Para caracterizar esta complejidad multidimensional, se requieren técnicas estadísticas que trasciendan el análisis univariado clásico. En este sentido, \citet{eisenstein2023identification} han demostrado que la variabilidad de los impactos extremos en ciclones puede reducirse a un número limitado de modos recurrentes utilizando técnicas de descomposición ortogonal.
% Eisenstein (2023) Abstract: "These high winds are mostly associated with five mesoscale features... The timing within the cyclone's life cycle... [varies]." (Valida que los ciclones complejos pueden descomponerse en "features" o modos principales).
Siguiendo este principio, esta tesis implementa la metodología de Funciones Ortogonales Empíricas Multivariadas (MEOF), fundamentada matemáticamente por \citet{liang2018multivariate}. A diferencia de los EOF tradicionales, esta técnica permite extraer modos de variabilidad acoplados entre campos físicamente distintos, revelando la coherencia espacial oculta donde la circulación del viento y los núcleos de precipitación covarían sistemáticamente.
% Liang (2018) Abstract: "The MEOF approach allows for the extraction of coupled patterns of variability shared among the variables... constructed from nitrate, salinity, and potential temperature..." (Valida el uso de MEOF para encontrar patrones compartidos entre variables distintas).
La aplicación de este marco estadístico sobre las fases dinámicas previamente clasificadas permitirá construir "prototipos de impacto", sintetizando la estructura típica del ciclón para cada región de génesis y estadio evolutivo, superando así la visión estática de los promedios climatológicos.

% ---% ---% ---% ---% ---% ---% ---% ---% ---% ---% ---% ---% ---% ---% ---% ---% ---% ---% ---% ---
% ---% ---% ---% ---% ---% ---% ---% ---% ---% ---% ---% ---% ---% ---% ---% ---% ---% ---% ---% ---
% --- PÁRRAFO 8: DATOS Y HERRAMIENTAS (ERA5 Y TRACKING POR VORTICIDAD) ---

La viabilidad de esta caracterización estructural reside en la reciente disponibilidad del reanálisis ERA5 \citep{hersbach2020era5}, cuya alta resolución espacial ($\sim$31 km) y temporal (horaria) permite resolver los gradientes de presión y los flujos de humedad que generaciones anteriores de datos suavizaban. Validaciones específicas para el Atlántico Sur realizadas por \citet{gramcianinov2020analysis} confirman que ERA5 ofrece una representación superior de la intensidad de los vientos superficiales y la estructura de los ciclones en comparación con productos previos como CFSR.
% Gramcianinov (2020): Validates ERA5 superiority in the South Atlantic.

Para capitalizar esta precisión, este estudio utiliza la base de datos de trayectorias de \citet{coutodesouza2024new}, la cual se fundamenta en el criterio de la vorticidad relativa en 850 hPa. Esta elección metodológica posee un consenso transversal en la literatura: establecida clásicamente por \citet{sinclair1994objective} y consolidada por \citet{padilhareinke2024objective} como el estándar más robusto para el Hemisferio Sur, este criterio es incluso utilizado en estudios recientes del Hemisferio Norte \citep{chen2025characteristics}. 
% Couto de Souza (2024): Database provider.
% Sinclair (1994) & Padilha Reinke (2024): Theoretical justification (Classic + SH context).
% % Padilha Reinke (2024) Chen (2025) & Han (2024): Modern validation (NH context), showing global methodology consensus.
A diferencia de los algoritmos basados exclusivamente en la presión mínima, el tracking por vorticidad permite el análisis del ciclo de vida de los eventos, capturando incluso aquellos sistemas rápidos que los algoritmos isobáricos tienden a omitir. Esta combinación de datos de alta fidelidad con una detección dinámica sensible constituye la base técnica necesaria para alimentar el análisis multivariado (MEOF) propuesto, permitiendo cuantificar finalmente la evolución de la huella espacial de los eventos compuestos en la región.
% ---% ---% ---% ---% ---% ---% ---% ---% ---% ---% ---% ---% ---% ---% ---% ---% ---% ---% ---% ---
% ---% ---% ---% ---% ---% ---% ---% ---% ---% ---% ---% ---% ---% ---% ---% ---% ---% ---% ---% ---
\section{Objetivos del Proyecto de Investigación}

\subsection{Objetivo General}
Caracterizar la estructura interna y la covariabilidad espacial de los campos de viento y precipitación en los ciclones extratropicales del Atlántico Sur a lo largo de su ciclo de vida, integrando la clasificación por fases evolutivas con análisis multivariado para establecer prototipos climáticos de impacto diferenciados por región de génesis.

\subsection{Objetivos Específicos}
\begin{enumerate}
    \item \textbf{Consolidar una base de datos lagrangiana} que asocie los campos de superficie de alta resolución (ERA5) con las trayectorias y fases dinámicas (incipiente, intensificación, madurez, decaimiento) identificadas para las tres regiones de ciclogénesis (SBR, LPB, ARG).
    \item \textbf{Determinar la asimetría espacial de los extremos} mediante la estimación de densidad por núcleos (KDE), cuantificando la probabilidad de ocurrencia de máximos de viento y precipitación respecto al centro del ciclón para identificar el desacoplamiento físico entre la presión mínima y la severidad en superficie.
    \item \textbf{Identificar los modos de variabilidad acoplada} aplicando Funciones Ortogonales Empíricas Multivariadas (MEOF) a los campos conjuntos de viento y precipitación, para revelar las estructuras coherentes de mesoescala que gobiernan los eventos compuestos en cada fase evolutiva.
    \item \textbf{Construir modelos conceptuales (prototipos)} para cada región de génesis, sintetizando los patrones dominantes obtenidos (MEOF) y las distribuciones de probabilidad de extremos (KDE) en una representación compuesta que describa la evolución típica del impacto desde la génesis hasta la lisis.
\end{enumerate}
