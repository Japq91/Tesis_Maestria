\chapter{Marco Teórico}\label{ch:marco_teorico}
%%%%%%%%
\section{Fundamentos físicos de los sistemas extratropicales}\label{sec:mt_fundamentos}


\subsection{Definición termodinámica y Espacio de Fase}\label{subsec:mt_fase}
La clasificación rigurosa de los sistemas de baja presión se fundamenta en su estructura térmica vertical y simetría. El marco teórico del **Espacio de Fase (Cyclone Phase Space)**, propuesto por \citet{hart2003cyclone}, establece que un ciclón extratropical es un vórtice fundamentalmente asimétrico de núcleo frío ($Cold-Core$).
Físicamente, esto implica que la intensidad del viento aumenta con la altura (por la relación del viento térmico) y que su fuente de energía primaria es la conversión baroclínica de energía potencial disponible en cinética \citep{hoskins2002new}. Esta definición teórica permite diferenciar los procesos dinámicos de latitudes medias de aquellos puramente convectivos (tropicales) o híbridos (subtropicales) que coexisten en el Atlántico Sur \citep{evans2012climatology}.

\subsection{Modelos de estructura y asimetría del viento}\label{subsec:mt_shapiro}
La distribución espacial de las variables en superficie responde a la evolución estructural del sistema. Como se detalla en la revisión teórica de \citet{coutodesouza2024thesis} (ver su Figura 2.4), existen dos paradigmas dominantes en la literatura: la evolución clásica hacia la oclusión y el modelo de **fractura frontal**.
Para los ciclones oceánicos intensos, revisiones recientes de \citet{schultz2021antecedents} confirman que la fractura del frente frío y el aislamiento del núcleo cálido (*warm seclusion*) son los mecanismos que gobiernan la distribución de impactos. Esta configuración estructural genera una asimetría inherente en el campo de vientos, donde los máximos de cantidad de movimiento (como los *sting jets*) se desacoplan del mínimo de presión, concentrándose en el sector oeste/noroeste del sistema \citep{clark2005sting}.

\section{Mecanismos forzantes de la ciclogénesis regional}\label{sec:mt_mecanismos}
La teoría de la ciclogénesis en el Hemisferio Sur establece que la formación de vórtices no es aleatoria, sino que responde a forzantes físicos estacionarios que actúan sobre el flujo de los oestes.

\subsection{Ciclogénesis de Sotavento (Forzante Orográfico)}\label{subsec:mt_lee}
El mecanismo físico dominante en latitudes medias de Sudamérica es la **ciclogénesis de sotavento** (\textit{Lee Cyclogenesis}). Según la teoría de conservación de la vorticidad potencial descrita por \citet{gan1994influence}, cuando una columna de aire atraviesa una barrera orográfica (los Andes), debe comprimirse verticalmente al subir y estirarse al bajar.
Este estiramiento vertical en el flanco oriental (sotavento) genera, por conservación del momento angular, un aumento de la vorticidad ciclónica relativa. Este proceso dinámico es la causa teórica de la alta frecuencia de formación de sistemas explosivos en la región de La Plata y Uruguay \citep{gramcianinov2019properties}.

\subsection{Ciclogénesis Diabática (Forzante Térmico)}\label{subsec:mt_diabatico}
En contraste con la dinámica orográfica, la ciclogénesis en latitudes subtropicales (costa sur de Brasil) se sustenta teóricamente en forzantes diabáticos.
Como explican \citet{reboita2021from}, la inestabilidad de la capa límite marina, generada por los fuertes flujos de calor latente y sensible sobre la Corriente de Brasil, actúa reduciendo la estabilidad estática efectiva de la atmósfera. Esto permite que perturbaciones débiles en altura se acoplen con la superficie y se intensifiquen rápidamente, un mecanismo energético distinto al forzamiento puramente baroclínico de las altas latitudes \citep{coutodesouza2024thesis}.

\section{Fundamentos del análisis de eventos compuestos y espacial}\label{sec:mt_teoria_analisis}

\subsection{Teoría de Eventos Compuestos}\label{subsec:mt_compound_theory}
El concepto de **Evento Compuesto** (\textit{Compound Event}) define una clase de fenómenos extremos donde el impacto resulta de la interacción estadística y física de múltiples variables, y no de sus valores individuales extremos \citep{zscheischler2018future}.
En la física de ciclones, \citet{chen2025characteristics} establecen que existe una dependencia no lineal entre la termodinámica (precipitación/calor latente) y la dinámica (viento/vorticidad). Por tanto, el marco teórico adecuado para su estudio no es univariado, sino que requiere evaluar la probabilidad conjunta de ocurrencia, dado que la retroalimentación entre ambas variables define la severidad del sistema.

\subsection{Estimación de Densidad y Variabilidad Espacial}\label{subsec:mt_kde_theory}
Dado que los ciclones extratropicales son sistemas móviles y asimétricos, los promedios eulerianos (en puntos fijos) tienden a suavizar las estructuras extremas.
Teóricamente, la distribución de impactos se describe mejor mediante funciones de densidad de probabilidad (PDF) en un marco de referencia lagrangiano (móvil). La **Estimación de Densidad por Kernel (KDE)** es la herramienta matemática no paramétrica que permite reconstruir la forma de esta distribución subyacente sin asumir normalidad \citep{priestley2022improved}.
Asimismo, para descomponer la complejidad de estos campos espaciales, la teoría de **Funciones Ortogonales Empíricas (EOF)** postula que la variabilidad de un campo atmosférico turbulento puede descomponerse en modos ortogonales de varianza máxima. Esto permite separar teóricamente las estructuras físicas coherentes (señal) del ruido de fondo, identificando patrones de acoplamiento físico entre variables distintas (MEOF) \citep{hannachi2007empirical}.