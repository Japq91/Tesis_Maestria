\chapter{Metodología}\label{ch:metodologia}

\section{Diseño general del estudio}\label{sec:diseno_general}

El objetivo metodológico central de esta tesis consiste en caracterizar la distribución espacial y la magnitud de los máximos de precipitación y viento asociados a los ciclones extratropicales del Atlántico Sur, adoptando un marco de referencia lagrangiano centrado en el sistema. El análisis se condiciona sistemáticamente por dos factores determinantes: la fase del ciclo de vida del ciclón y su región de ciclogénesis en Sudamérica.

La estrategia experimental se estructura en una secuencia lógica: se parte de una base de datos preexistente de trayectorias y una clasificación objetiva de fases evolutivas; sobre esta base, se extraen campos horarios del reanálisis alrededor del centro de cada sistema para construir compositos representativos y, finalmente, se aplican técnicas estadísticas para cuantificar los extremos de las variables de impacto.

\section{Datos del Reanálisis ERA5}\label{sec:datos_era5}

Para la caracterización de los campos atmosféricos se utilizó el reanálisis de quinta generación del ECMWF, ERA5, descrito por \citet{hersbach2020era5}. Se analizó el período comprendido entre 1979 y 2020, accediendo específicamente a las variables de precipitación y viento a través del conjunto de datos \textit{``ERA5 hourly data on single levels from 1940 to present''} \citep{hersbach_era5_2023b}.

La elección de ERA5 responde a su alta resolución espacial ($\sim$31 km) y temporal (horaria). Esta capacidad para resolver escalas finas es crítica en este estudio, ya que, tal como han demostrado \citet{priestley2022improved} y \citet{dalanhese2023new}, la mayor resolución reduce la subestimación de la intensidad de los sistemas observada en generaciones previas de reanálisis, permitiendo una representación más realista de los gradientes de viento y los núcleos de precipitación asociados a los ciclones.

\subsection{Variables meteorológicas}\label{subsec:variables}

La caracterización de los patrones espaciales se realizó a partir de campos horarios extraídos de ERA5, seleccionando las variables que describen la intensidad hidrometeorológica y dinámica del sistema:
\begin{itemize}
    \item Precipitación total acumulada.
    \item Magnitud del viento en superficie (10 y 100 m).
\end{itemize}

La decisión de analizar conjuntamente los campos de viento y precipitación responde a la necesidad de capturar la estructura de los eventos compuestos (\textit{compound events}). Investigaciones recientes en latitudes medias, como la de \citet{chen2025characteristics}, destacan que la co-ocurrencia espacial y temporal de máximos de viento y precipitación es una propiedad intrínseca de la dinámica de los ciclones extratropicales. Esta interacción define el evento más allá de sus componentes individuales, lo que ayuda a justificar la aplicación de técnicas multivariadas para desacoplar sus modos de variabilidad y entender su coherencia espacial.

\section{Base de datos de ciclones}\label{sec:bd_ciclones_fases}

El insumo principal de esta investigación es una base de datos climatológica integrada que combina la detección lagrangiana de los sistemas con una caracterización objetiva de su evolución dinámica. Específicamente, se utilizó la base de datos de ciclones del Atlántico Sur generada y procesada por \citet{coutodesouza2024new}, la cual se construye sobre el catálogo de trayectorias originalmente desarrollado por \citet{gramcianinov2019properties}.

\subsection{Trayectorias y detección}\label{subsec:origen_tracking}

El componente de seguimiento (\textit{tracking}) se basa en la climatología de \citet{gramcianinov2019properties}, desarrollada a partir del reanálisis ERA5 utilizando el algoritmo TRACK para la identificación de los sistemas usando el campo de vorticidad relativa en 850 hPa ($\zeta_{850}$); los detalles técnicos sobre la configuración específica del algoritmo y el filtrado espectral utilizado para aislar la escala sinóptica se encuentran descritos en profundidad en \citet{gramcianinov2019properties} y \citet{hoskins2005new}.

La elección de trabajar con trayectorias derivadas de la vorticidad, en lugar de la presión al nivel del mar (MSLP), se fundamenta en la recomendación metodológica de \citet{sinclair1994objective}. Este autor estableció que la vorticidad evita el sesgo sistemático hacia ciclones lentos y profundos, permitiendo capturar adecuadamente los sistemas móviles típicos de las latitudes medias. Además, la base de datos utilizada incorpora las ventajas del reanálisis ERA5 en regiones de orografía compleja: según \citet{gramcianinov2020analysis}, su alta resolución permite una detección más temprana y precisa de centros ciclónicos al sotavento de los Andes. Cabe destacar que el catálogo original de \citet{gramcianinov2019properties} ya incluye filtros de relevancia sinóptica, conservando únicamente aquellos sistemas con una duración mínima de 24 horas y un desplazamiento superior a los 1000 km.

\subsection{Fases del ciclo de vida}\label{subsec:bd_enriquecida}

Para los fines de esta tesis, no se utilizaron las trayectorias crudas, sino la versión enriquecida por \citet{coutodesouza2024new}. Estos autores procesaron el catálogo original aplicando el algoritmo \textit{CycloPhaser} \citep{deSouza2025CycloPhaser} para segmentar objetivamente la historia de cada ciclón. Como resultado, la base de datos final utilizada en este estudio proporciona, para cada paso de tiempo, la posición del sistema y su fase evolutiva correspondiente, definida según la tasa de cambio de la vorticidad central:

\begin{enumerate}
    \item \textbf{Incipiente (Ic):} Etapa inicial de organización del vórtice.
    \item \textbf{Intensificación (It):} Período de rápido aumento (negativo) de la vorticidad ciclónica.
    \item \textbf{Madurez (M):} Etapa de máxima intensidad y estabilización relativa.
    \item \textbf{Decaimiento (D):} Fase de debilitamiento progresivo (ciclolisis).
\end{enumerate}

%\subsection{Selección del subconjunto}\label{subsec:filtro_lineal}
\subsection{Selección de ciclos de vida canónicos}\label{subsec:ciclos_canonicos}
A partir de esta base de datos procesada, se aplicó un criterio de selección específico para el análisis estadístico de los impactos. Si bien la transición entre fases puede presentar re-intensificaciones complejas, esta investigación se restringe a un subconjunto de ciclones con una evolución lineal.

Esto implica que se seleccionaron únicamente los sistemas que atraviesan estrictamente y en orden las cuatro fases: Incipiente $\rightarrow$ Intensificación $\rightarrow$ Madurez $\rightarrow$ Decaimiento. Este arquetipo representa aproximadamente el 60\% de los sistemas del Atlántico Sur según las estadísticas de \citet{coutodesouza2024new}. Esta restricción metodológica es necesaria para garantizar la homogeneidad de la muestra, asegurando que los compositos de cada fase (Ic, It, M, D) sean físicamente comparables y no estén contaminados por la redistribución de energía asociada a ciclos de vida no monotónicos.

\section{Procesamiento}\label{sec:procesamiento_lagrangiano}

Para analizar la estructura de los ciclones independientemente de su ubicación geográfica absoluta, se adopta un marco de referencia relativo o lagrangiano. Para cada instante $t$ de la vida de un ciclón, se define un dominio móvil de $20^{\circ} \times 20^{\circ}$ centrado en las coordenadas de su núcleo, proporcionadas por la base de datos de trayectorias. Dentro de este dominio, se extraen los campos de las variables de impacto.

Este enfoque de composición centrada en el ciclón es fundamental para aislar la señal de mesoescala de los sistemas extratropicales y ha sido ampliamente utilizado para caracterizar sus entornos en el Atlántico Sur \citep{crespo2020potential, priestley2022improved}. El análisis se realiza de forma separada para cada combinación de región de ciclogénesis y fase del ciclo de vida, permitiendo estratificar los resultados de manera simultánea por estos dos factores.

Para evitar la variabilidad de alta frecuencia inherente a los datos horarios y obtener un campo representativo de la estructura media del ciclón en cada fase, se aplica un promediado temporal. Específicamente, para cada ciclón y para cada fase por la que atraviesa, se seleccionan los dos o tres tiempos centrales de dicha fase (dependiendo de si su duración en horas es par o impar) y se promedian sus campos espaciales. Este procedimiento genera un campo promedio por ciclón--fase, sobre el cual se realizarán los análisis subsiguientes.
%%%%%%%%%%% # %%%%%%%%%%% # %%%%%%%%%% # %%%%%%%%%%% # %%%%%%%%%% # %%%%%%%%%%% # %%%%%%%%%%

\section{Estratificación por regiones de ciclogénesis}\label{sec:regiones_ciclogenesis}

La base de datos incluye la localización de la génesis de cada sistema, permitiendo estratificar el análisis geográficamente. Para la definición de los dominios, este estudio adopta estrictamente la taxonomía espacial propuesta por \citet{gramcianinov2019properties}, la cual identifica tres regiones en Sudamérica basadas en la densidad de ciclogénesis:
% Gramcianinov (2019): Identifies three main genesis density hotspots: one over the coast of Southern Brazil, one over La Plata basin/Uruguay, and one over Southern Argentina.

\begin{itemize}
    \item SBR (Sur-Sudeste de Brasil): $52^\circ-38^\circ$W, $30^\circ-20^\circ$S
    \item LPB (La Plata/Uruguay): $69^\circ-52^\circ$W, $38^\circ-23^\circ$S
    \item ARG (Patagonia/Argentina): $70^\circ-50^\circ$W, $55^\circ-39^\circ$S
\end{itemize}

Esta segmentación operativa es consistente con los regímenes dinámicos descritos en la síntesis climática reciente de \citet{reboita2026meteorology}. Según esta revisión, cada sector presenta mecanismos de forzamiento únicos:
La región SBR se distingue por un régimen frecuentemente influenciado por procesos diabáticos y transiciones híbridas. Como detallan \citet{marrafon2022classificacao} y \citet{reboita2022from}, es aquí donde la interacción con los gradientes de la Temperatura Superficial del Mar (TSM) y el transporte de humedad favorece el desarrollo de sistemas con núcleos cálidos. Por su parte, la región LPB actúa como una zona de transición crítica. Estudios de \citet{crespo2020potential} demuestran que la ciclogénesis en este sector está fuertemente forzada por anomalías de vorticidad potencial en niveles altos, lo que, sumado a la baroclinicidad costera, favorece una alta densidad de eventos explosivos \citep{andrade2024composite}. Finalmente, ARG representa el entorno clásico dominado por la inestabilidad del frente polar y el forzante orográfico de los Andes \citep{vera2002cold, inatsu2004zonal}.

%%%%%%%%%%% # %%%%%%%%%%% # %%%%%%%%%%% # %%%%%%%%%%% # %%%%%%%%%%% # %%%%%%%%%%% # %%%%%%%%%%
\section{Técnicas estadísticas de caracterización}\label{sec:tecnicas_estadisticas}

\subsection{Extracción de máximos y análisis de histogramas}\label{subsec:extraccion_histogramas}

A partir del campo promedio por ciclón--fase, se identifica la magnitud y la ubicación (en coordenadas relativas al centro) del valor máximo de la variable dentro del dominio de $20^{\circ} \times 20^{\circ}$. Este procedimiento genera, para cada combinación región--fase--variable, un conjunto de $m$ valores máximos (uno por ciclón).

Con estos valores se construyen histogramas de frecuencias para describir la distribución de las magnitudes máximas. Este análisis permite comparar cuantitativamente cómo cambian los rangos de valores más probables o extremos entre las diferentes fases del ciclo de vida y entre las distintas regiones de ciclogénesis.

\subsection{Estimación de densidad de kernel (KDE) de máximos}\label{subsec:kde_localizacion}

Para caracterizar la distribución espacial preferencial de la ocurrencia de máximos, se transforma cada campo promedio por ciclón--fase en un mapa binario. En este mapa, todos los píxeles tienen valor 0, excepto el píxel correspondiente a la ubicación del máximo, que tiene valor 1. A partir del conjunto de $m$ mapas binarios para una combinación región--fase--variable dada, se aplica una estimación de densidad de kernel (KDE).

El resultado es un campo continuo de densidad de probabilidad relativa que indica, para cada posición dentro del dominio centrado, la probabilidad de encontrar el máximo de la variable. Esta técnica es particularmente útil para visualizar asimetrías sistemáticas en la distribución espacial de los impactos, permitiendo evaluar, por ejemplo, si los vientos máximos tienden a ubicarse en un cuadrante específico respecto al centro, y cómo esto varía con la fase del ciclo de vida \citep{gramcianinov2023impact}.

\subsection{Análisis de funciones ortogonales empíricas (EOF) univariado}\label{subsec:eof_univariado}

Para identificar los patrones espaciales dominantes de variabilidad en la estructura de los ciclones dentro del marco centrado, se aplica un análisis de Funciones Ortogonales Empíricas (EOF). Este análisis se realiza de manera separada para cada variable (V10, V100, precipitación), región de ciclogénesis y fase del ciclo de vida, utilizando el conjunto de $m$ campos promedio (compositos) por ciclón--fase.

Previo al análisis, a cada campo composito se le resta la media espacial del conjunto y se aplica una ponderación por la raíz cuadrada del coseno de la latitud para tener en cuenta la convergencia de los meridianos. La EOF descompone la varianza total del conjunto en modos espaciales ortogonales (las EOFs) y sus correspondientes series temporales (componentes principales). Este método, adaptado al dominio lagrangiano, permite aislar y comparar las estructuras espaciales más recurrentes o típicas asociadas a los ciclones en cada contexto (región/fase), siguiendo la lógica de análisis de patrones aplicada en estudios sinópticos \citep{vera2002cold}.

\subsection{EOF combinado (multivariado) precipitación--viento}\label{subsec:eof_multivariado}

Para evaluar la covariabilidad espacial entre los campos de precipitación y viento, se realiza un análisis EOF combinado (MEOF). Dado que estas variables poseen unidades físicas y rangos de varianza dispares (mm/h vs. m/s), es un requisito metodológico fundamental estandarizar las series antes de su integración para evitar que la variable con mayor magnitud numérica domine los modos resultantes \citep{wilks2019statistical}.

El procedimiento consiste en: (i) calcular las anomalías de cada campo respecto a su media; (ii) normalizar dichas anomalías dividiéndolas por la desviación estándar espacialmente promediada de cada variable, asegurando un peso equiparable en la matriz de covarianza combinada; y (iii) concatenar espacialmente los campos normalizados para formar un vector de estado ampliado, siguiendo el esquema de análisis de patrones acoplados descrito por \citet{bretherton1992intercomparison}.

El análisis EOF aplicado a esta matriz combinada permite obtener modos acoplados que expliquen la varianza conjunta, siguiendo la base matemática de descomposición multivariada validada por \citet{liang2018multivariate}. La aplicación de esta técnica es instrumental para responder a interrogantes físicas recientes planteadas por \citet{sinclair2023relationship}, permitiendo revelar si los núcleos de precipitación intensa se co-ubican sistemáticamente con los gradientes de viento más fuertes y cómo esta coherencia estructural evoluciona a lo largo del ciclo de vida.

\section{Análisis de sensibilidad: subconjunto de ciclones intensos}\label{sec:ciclones_intensos}

Con el fin de evaluar si los patrones identificados cambian para sistemas con vorticidad mas intensa y potencialmente más dañinos, se define un subconjunto de ciclones basado en un criterio dinámico. Para cada ciclón en la base de datos, se extrae el valor mínimo de vorticidad relativa a 850 hPa ($\zeta_{850}^{min}$) a lo largo de todo su ciclo de vida. En el Hemisferio Sur, valores más negativos indican un vórtice más intenso. El subconjunto intenso se define como el 10\% de ciclones con los valores de $\zeta_{850}^{min}$ más negativos. 
Enfoques previos, como \citet{reboita2010south}, han empleado criterios dinámicos basados en vorticidad para identificar y discriminar ciclones según su intensidad.

Sobre este subconjunto se repite íntegramente el flujo metodológico descrito en las secciones anteriores: construcción de campos promedio por fase, extracción de máximos, análisis de histogramas, KDE de localización y análisis EOF (tanto univariado como multivariado). La comparación sistemática entre los resultados del conjunto total y los del subconjunto intenso permite evaluar de manera robusta si una mayor intensidad dinámica del vórtice se asocia con cambios significativos en: (i) la magnitud de los máximos de viento y precipitación, (ii) la distribución espacial relativa de estos máximos, y (iii) los patrones estructurales dominantes, tanto para cada fase como para cada región de ciclogénesis. Estudios compositivos recientes sugieren que los ciclones más intensos, especialmente los explosivos, pueden presentar estructuras frontales y patrones de impacto distintivos \citep{andrade2024composite}.